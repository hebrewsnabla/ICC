\documentclass[10pt,aspectratio=43]{beamer}
%\usepackage{nju}			 %导入 nju 模板宏包
%\usepackage[UTF8]{ctex}      %导入 ctex 宏包,添加中文支持
\usepackage{xeCJK}
\usepackage{amsmath,amsfonts,amssymb,bm}   %导入数学公式所需宏包
\usepackage{color}			 %字体颜色支持
\usepackage{graphicx,hyperref,url}
\usepackage{listings}
\usepackage{booktabs}
\usepackage{multirow}
\usepackage{float}
\usepackage{mhchem}
\usepackage{animate}
\usepackage[isbn=false,doi=false,url=false,uniquename=init,style=chem-acs]{biblatex}
\addbibresource{icc_pre_1127.bib}
\AtEveryCitekey{\clearfield{title}}
\AtEveryBibitem{\clearfield{title}}
\usepackage{textcomp}
\usepackage{multicol}

\usepackage{listings}
\definecolor{dkgreen}{rgb}{0,0.6,0}
\definecolor{gray}{rgb}{0.5,0.5,0.5}
\definecolor{mauve}{rgb}{0.58,0,0.82}
\lstset{ %
	language=Python,                % the language of the code
	basicstyle=\Consolas\footnotesize,           % the size of the fonts that are used for the code
	numbers=left,                   % where to put the line-numbers
	numberstyle=\tiny\color{gray},  % the style that is used for the line-numbers
	stepnumber=1,                   % the step between two line-numbers. If it's 1, each line 
	% will be numbered
	numbersep=5pt,                  % how far the line-numbers are from the code
	backgroundcolor=\color{white},      % choose the background color. You must add \usepackage{color}
	showspaces=false,               % show spaces adding particular underscores
	showstringspaces=false,         % underline spaces within strings
	showtabs=false,                 % show tabs within strings adding particular underscores
	frame=single,                   % adds a frame around the code
	rulecolor=\color{black},        % if not set, the frame-color may be changed on line-breaks within not-black text (e.g. commens (green here))
	tabsize=4,                      % sets default tabsize to 2 spaces
	captionpos=b,                   % sets the caption-position to bottom
	breaklines=true,                % sets automatic line breaking
	breakatwhitespace=false,        % sets if automatic breaks should only happen at whitespace
	title=\lstname,                   % show the filename of files included with \lstinputlisting;
	% also try caption instead of title
	keywordstyle=\color{blue},          % keyword style
	commentstyle=\color{dkgreen},       % comment style
	stringstyle=\color{mauve},         % string literal style
	escapeinside={\%*}{*)},            % if you want to add LaTeX within your code
	morekeywords={*,...}               % if you want to add more keywords to the set
}
\usepackage{multicol}
\usepackage{multirow}
\usepackage{siunitx}
\definecolor{codegray}{gray}{0.9}
\newfontfamily\Consolas{Consolas}
\newcommand{\code}[1]{\colorbox{codegray}{{\Consolas#1}}}
\newcommand{\hl}[1]{\colorbox{yellow}{#1}}
\lstnewenvironment{mcode}
{\lstset{backgroundcolor=\color{lightgray},
		xleftmargin=0.5cm,
		frame=tlbr,framesep=4pt,framerule=0pt,
		%language=python,
		keepspaces=false,
		numbers=none
}}%
{}

\usepackage{tikz,mathpazo}
\usetikzlibrary{shapes.geometric, arrows}
\usetikzlibrary{calc}
\tikzstyle{box} = [rectangle, minimum width = 1.5cm, minimum height=0.8cm,text centered, draw = black]
\tikzstyle{rbox} = [rectangle, rounded corners, minimum width = 1.5cm, minimum height=0.8cm,text centered, draw = black, fill = blue!40]
\tikzstyle{arrow} = [->,>=stealth]

\usepackage{braket}
\usepackage{physics}

%\beamertemplateballitem		%设置 Beamer 主题
%\catcode`\。=\active        %或者=13
%\newcommand{。}{.}         %将正文中的“。”号转换为“.”。

%\AtBeginSection[]
%{
%  \begin{frame}
%    \frametitle{Contents}
%    \tableofcontents[currentsection]
%  \end{frame}
%}

\numberwithin{equation}{section}

\title{Charge Decomposition Analysis}	        %首页信息设置

\author[]{            %个人信息设置
    Shirong Wang\\[0.3cm]
    %15XXXXXXXX\\[0.3cm]
    Kuang Yaming Honors School}

\date{\today}



\begin{document}
\begin{frame}
%\hfill ICC Report III
\titlepage
\end{frame}

\begin{frame}
\frametitle{Contents}
\tableofcontents
\end{frame}

\section{Methodology}
\begin{frame}
\frametitle{Charge Decomposition Analysis}
Charge Decomposition Analysis was proposed by S. Dapprich and G. Frenking, to analyze charge transfer.\\
\hfill \fullcite{Dapprich1995}
~\vspace{40pt}\\
Common approaches for analyzing charge transfer
\begin{enumerate}
	\item Atomic charges
	\item Density difference
	\item Charge decomposition analysis
	\item Energy decomposition analysis
\end{enumerate}
\end{frame}

\begin{frame}
\frametitle{}
Expand each MO with N AOs (basis)
\begin{equation}\label{key}
\phi_i = \sum_k^N C_{ki}\chi_k
\end{equation}
\begin{table}
	\centering
	\begin{tabular}{ccccc}
		\hline
		& math variable & subscript & number & number example\\ \hline
		MO & $ \phi $ & $ i $ & N & 420\\
		AO & $ \chi $ & $ k $ & N & 420\\
		occupied MO &   &   &   occ & 52\\
		virtual MO &   &     &  vir & 368\\
		%MO Coeff & C   & $ k,i $ & $ N\cross N $ & 420$ \cross $420\\
		\hline
	\end{tabular}
\end{table}
$ N $ is determined by user, e.g. \code{def2TZVP}.\\
$ occ = K/2 $, where $ K $ is the number of electrons (In restricted case).\\

\end{frame}


\defverbatim[colored]\makeset{
\begin{mcode}
 Standard basis: def2TZVP (5D, 7F)
Ernie: Thresh=  0.10000D-02 Tol=  0.10000D-05 Strict=F.
There are   466 symmetry adapted cartesian basis functions of A   symmetry.
There are   420 symmetry adapted basis functions of A   symmetry.
420 basis functions,   679 primitive gaussians,   466 cartesian basis functions
52 alpha electrons       52 beta electrons
...
NBasis=   420 RedAO= T EigKep=  6.59D-04  NBF=   420
NBsUse=   420 1.00D-06 EigRej= -1.00D+00 NBFU=   420
\end{mcode}
}
\begin{frame}
%\frametitle{Basis}
check \code{.log} file
\makeset
\end{frame}

\begin{frame}
\frametitle{MO Coefficient and Overlap Matrix}
\begin{equation}\label{key}
K/2 = \sum_i^{occ}\Braket{\phi_i | \phi_i} = \sum_i^{occ} \sum_k^N\sum_l^N C_{ki}^*C_{li} \Braket{\chi_k | \chi_l}
\end{equation}
or
\begin{align}\label{key}
K = \sum_i^N \eta_i\Braket{\phi_i | \phi_i} = \sum_i^N \eta_i \sum_k^N\sum_l^N C_{ki}^*C_{li} \Braket{\chi_k | \chi_l} 
\end{align}
where 
\begin{equation}\label{key}
\eta_i = 
\left\{
\mqty{2 & i\in occ\\ 0 & i\in vir}
\right.
\end{equation}
Moreover, define
\begin{align}
\text{Density Matrix} & & P_{k\ell} = \sum_i^N \eta_i C_{ki}^* C_{\ell i} \\
\text{Overlap Matrix} & & S_{k\ell} = \Braket{\chi_k | \chi_\ell}
\end{align}
thus
\begin{equation}\label{key}
K = \sum_k^N \sum_\ell^N P_{k\ell} S_{k\ell}
\end{equation}

\end{frame}



\begin{frame}
\frametitle{Fragment Orbitals}
Recall that
\begin{equation}\label{key}
\Braket{\phi_i | \phi_i} = \sum_k^N \sum_\ell^N C_{ki}^* C_{\ell i} S_{kl}
\end{equation}
where $ N $ is the number of AOs, and $ S $ is the overlap matrix between AOs. \\
Do SCF calculation for two fragments at same geometry separately, we get two set of MOs of fragments. They are called fragment orbitals, with a total number $ N $.\\
Expand total MO with FOs instead, we get
\begin{equation}\label{key}
\eta_i\Braket{\phi_i | \phi_i} = \eta_i\sum_m^N \sum_n^N C_{mi}^* C_{ni} S_{mn}
\end{equation}
where
\begin{align}\label{key}
S_{mn} = \Braket{\phi_m | \phi_n}
\end{align}

\end{frame}

\begin{frame}
Define
\begin{align}
d_i = \sum_{m\in A}^{occ} \sum_{n\in B}^{vir} \eta_i C_{mi}^* C_{ni} S_{mn}\\
b_i = \sum_{m\in A}^{vir} \sum_{n\in B}^{occ} \eta_i C_{mi}^* C_{ni} S_{mn}\\
r_i = \sum_{m\in A}^{occ} \sum_{n\in B}^{occ} \eta_i C_{mi}^* C_{ni} S_{mn}\\
d = \sum_i d_i \qquad b = \sum_i b_i \qquad r = \sum_i r_i
\end{align}
thus\\
$ d $ is the charge transfer from A to B, $ b $ is the charge transfer from B to A. $ d-b $ is the net charge transfer from A to B.

\end{frame}

\section{Technical Details}
\begin{frame}
\frametitle{Technical Details}
If use \code{.log} file
\begin{itemize}
	\item use \code{nosymm} to prevent geometric transformation
	\item use \code{pop=full} to print all MO coefficients
	\item use \code{iop(3/33=1)} to print overlap matrix
\end{itemize}
Or, use \code{.fchk} file
\begin{itemize}
	\item \code{nosymm}
	\item calculate overlap matrix by self
\end{itemize}

\end{frame}

\begin{frame}
\frametitle{Technical Details}
Other tips:
\begin{itemize}
	\item Diffuse basis functions may destroy the result
	\item Atom coordinates must be arranged in the same order in complex and fragments
\end{itemize}
\end{frame}

\section{Generalized CDA}
\begin{frame}
\frametitle{What if $ \eta_i $ is not integer?}
In post-HF calculations, we can obtain non-integer occupied natural orbitals (NOs).\\
Tian Lu et al. proposed Generalized CDA
\begin{align}
t_i = \sum_{m\in A} \sum_{n\in B} \eta_i \dfrac{\eta_m - \eta_n}{\eta_{\text{ref}}}C_{mi}^* C_{ni}S_{mn} \\
r_i = \sum_{m\in A} \sum_{n\in B} \eta_i \dfrac{2\min(\eta_m, \eta_n)}{\eta_{\text{ref}}}C_{mi}^* C_{ni}S_{mn} 
\end{align}
where $ \eta_{\text{ref}} = 2 $ for closed-shell cases, and $ \eta_{\text{ref}}=1 $ for open-shell cases.\\
\hfill\fullcite{2015}


\end{frame}

\section{Expectations}
\begin{frame}
\frametitle{Expectations}
\begin{itemize}
	\item How to access FO-FO charge transfer value?
	\item FO Composition of MOs
	\begin{equation}\label{key}
	\Theta_{m,i} = \sum_n C_{mi}C_{ni}S_{mn}
	\end{equation}
	\item Is NBO works for CDA calculation?
\end{itemize}

\end{frame}


	
\end{document}