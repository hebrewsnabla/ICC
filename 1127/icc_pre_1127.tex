\documentclass[10pt,aspectratio=43]{beamer}
%\usepackage{nju}			 %导入 nju 模板宏包
%\usepackage[UTF8]{ctex}      %导入 ctex 宏包,添加中文支持
\usepackage{xeCJK}
\usepackage{amsmath,amsfonts,amssymb,bm}   %导入数学公式所需宏包
\usepackage{color}			 %字体颜色支持
\usepackage{graphicx,hyperref,url}
\usepackage{listings}
\usepackage{booktabs}
\usepackage{multirow}
\usepackage{float}
\usepackage{mhchem}
\usepackage{animate}
\usepackage[isbn=false,doi=false,url=false,uniquename=init,style=chem-acs]{biblatex}
\addbibresource{icc_pre_1108.bib}
\AtEveryCitekey{\clearfield{title}}
\AtEveryBibitem{\clearfield{title}}
\usepackage{textcomp}
\usepackage{multicol}

\usepackage{listings}
\definecolor{dkgreen}{rgb}{0,0.6,0}
\definecolor{gray}{rgb}{0.5,0.5,0.5}
\definecolor{mauve}{rgb}{0.58,0,0.82}
\lstset{ %
	language=Python,                % the language of the code
	basicstyle=\Consolas\footnotesize,           % the size of the fonts that are used for the code
	numbers=left,                   % where to put the line-numbers
	numberstyle=\tiny\color{gray},  % the style that is used for the line-numbers
	stepnumber=1,                   % the step between two line-numbers. If it's 1, each line 
	% will be numbered
	numbersep=5pt,                  % how far the line-numbers are from the code
	backgroundcolor=\color{white},      % choose the background color. You must add \usepackage{color}
	showspaces=false,               % show spaces adding particular underscores
	showstringspaces=false,         % underline spaces within strings
	showtabs=false,                 % show tabs within strings adding particular underscores
	frame=single,                   % adds a frame around the code
	rulecolor=\color{black},        % if not set, the frame-color may be changed on line-breaks within not-black text (e.g. commens (green here))
	tabsize=4,                      % sets default tabsize to 2 spaces
	captionpos=b,                   % sets the caption-position to bottom
	breaklines=true,                % sets automatic line breaking
	breakatwhitespace=false,        % sets if automatic breaks should only happen at whitespace
	title=\lstname,                   % show the filename of files included with \lstinputlisting;
	% also try caption instead of title
	keywordstyle=\color{blue},          % keyword style
	commentstyle=\color{dkgreen},       % comment style
	stringstyle=\color{mauve},         % string literal style
	escapeinside={\%*}{*)},            % if you want to add LaTeX within your code
	morekeywords={*,...}               % if you want to add more keywords to the set
}
\usepackage{multicol}
\usepackage{multirow}
\usepackage{siunitx}
\definecolor{codegray}{gray}{0.9}
\newfontfamily\Consolas{Consolas}
\newcommand{\code}[1]{\colorbox{codegray}{{\Consolas#1}}}
\newcommand{\hl}[1]{\colorbox{yellow}{#1}}
\lstnewenvironment{mcode}
{\lstset{backgroundcolor=\color{lightgray},
		xleftmargin=0.5cm,
		frame=tlbr,framesep=4pt,framerule=0pt,
		%language=python,
		keepspaces=false,
		numbers=none
}}%
{}

\usepackage{tikz,mathpazo}
\usetikzlibrary{shapes.geometric, arrows}
\usetikzlibrary{calc}
\tikzstyle{box} = [rectangle, minimum width = 1.5cm, minimum height=0.8cm,text centered, draw = black]
\tikzstyle{rbox} = [rectangle, rounded corners, minimum width = 1.5cm, minimum height=0.8cm,text centered, draw = black, fill = blue!40]
\tikzstyle{arrow} = [->,>=stealth]

\usepackage{braket}
\usepackage{physics}

%\beamertemplateballitem		%设置 Beamer 主题
%\catcode`\。=\active        %或者=13
%\newcommand{。}{.}         %将正文中的“。”号转换为“.”。

%\AtBeginSection[]
%{
%  \begin{frame}
%    \frametitle{Contents}
%    \tableofcontents[currentsection]
%  \end{frame}
%}

\numberwithin{equation}{section}

\title{Molecular Orbitals and Charge Decomposition Analysis}	        %首页信息设置

\author[]{            %个人信息设置
    Shirong Wang\\[0.3cm]
    %15XXXXXXXX\\[0.3cm]
    Kuang Yaming Honors School}

\date{\today}



\begin{document}
\begin{frame}
\hfill ICC Report III
\titlepage
\end{frame}

\begin{frame}
\frametitle{Contents}
\tableofcontents
\end{frame}

\section{Molecular Orbitals}
\begin{frame}
\frametitle{}
Expand each MO with N AOs (basis)
\begin{equation}\label{key}
\phi_i = \sum_k^N C_{ki}\chi_k
\end{equation}

\begin{tabbing}
	\centering
	\begin{tabular}{ccccc}
		\hline
		& math variable & subscript & number & number example\\ \hline
		MO & $ \phi $ & $ i $ & N & 420\\
		AO & $ \chi $ & $ k $ & N & 420\\
		occupied MO &   &   &   occ & 52\\
		virtual MO &   &     &  vir & 368\\
		%MO Coeff & C   & $ k,i $ & $ N\cross N $ & 420$ \cross $420\\
		\hline
	\end{tabular}
\end{tabbing}
$ N $ is determined by user, e.g. \code{def2TZVP}.\\
$ occ = K/2 $, where $ K $ is the number of electrons (In restricted case).\\

\end{frame}

\defverbatim[colored]\makeset{
\begin{mcode}
Title Card Required
SP        RPBE1PBE                             def2TZVP
Number of atoms                            I               29
	
Full Title                                 C   N=           2
Title Card Required
Route                                      C   N=           8
#p pbe1pbe/def2TZVP geom=allcheck scrf=(smd,solvent=benzene)
nosymm pop=full IOp(3/33=1,3/32=2)
Charge                                     I                0
Multiplicity                               I                1
Number of electrons                        I              104
Number of alpha electrons                  I               52
Number of beta electrons                   I               52
Number of basis functions                  I              420
Number of independent functions            I              420
\end{mcode}
}
\begin{frame}
\frametitle{But How Can I Count ...}
Check the \code{.fchk} file!
\makeset
\end{frame}

\defverbatim[colored]\makeset{
\begin{mcode}
 Standard basis: def2TZVP (5D, 7F)
Ernie: Thresh=  0.10000D-02 Tol=  0.10000D-05 Strict=F.
There are   466 symmetry adapted cartesian basis functions of A   symmetry.
There are   420 symmetry adapted basis functions of A   symmetry.
420 basis functions,   679 primitive gaussians,   466 cartesian basis functions
52 alpha electrons       52 beta electrons
...
NBasis=   420 RedAO= T EigKep=  6.59D-04  NBF=   420
NBsUse=   420 1.00D-06 EigRej= -1.00D+00 NBFU=   420
\end{mcode}
}
\begin{frame}
%\frametitle{Basis}
or \code{.log} file
\makeset
\end{frame}

\begin{frame}
\frametitle{MO Coefficient and Overlap Matrix}
\begin{equation}\label{key}
K/2 = \sum_i^{occ}\Braket{\phi_i | \phi_i} = \sum_i^{occ} \sum_k^N\sum_l^N C_{ki}^*C_{li} \Braket{\chi_k | \chi_l}
\end{equation}
or
\begin{align}\label{key}
K = \sum_i^N \eta_i\Braket{\phi_i | \phi_i} = \sum_i^N \eta_i \sum_k^N\sum_l^N C_{ki}^*C_{li} \Braket{\chi_k | \chi_l} 
\end{align}
where 
\begin{equation}\label{key}
\eta_i = 
\left\{
\mqty{2 & i\in occ\\ 0 & i\in vir}
\right.
\end{equation}
Moreover, define
\begin{align}
\text{Density Matrix} & & P_{k\ell} = \sum_i^N \eta_i C_{ki}^* C_{\ell i} \\
\text{Overlap Matrix} & & S_{k\ell} = \Braket{\chi_k | \chi_\ell}
\end{align}
thus
\begin{equation}\label{key}
K = \sum_k^N \sum_\ell^N P_{k\ell} S_{k\ell}
\end{equation}

\end{frame}

\section{Charge Decomposition Analysis}
\begin{frame}
\frametitle{Charge Decomposition Analysis}

\end{frame}

\begin{frame}
\frametitle{Fragment Orbitals}

\end{frame}
	
\end{document}